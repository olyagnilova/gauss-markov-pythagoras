\section{Introduction}

\begin{fullwidth}

Pursuing abstractness and generality, a number of books and articles in
econometrics rely heavily on standard algebraic proofs.
However, most of the theorems proved in such a way have a strong geometric appeal.
This paper demonstrates how shorter, less technical and even more beautiful the proofs
can be when they are based on geometric theorems.
We show that this technique can be extended to explain
concepts in probability and statistics.

Although most of the theorems and ideas are not completely new and
there are even a few works on the geometry in econometrics and statistics,
this paper introduces alternative explanations and more general results in some cases.
Further, there are algebraic proofs provided parallel to geometric ones
and illustrations which are our own work.
The illustrations are published at \url{https://github.com/olyagnilova/gauss-markov-pythagoras} and licensed under the
Creative Commons Attribution 4.0. % todo: wiki
They are free to use and have the potential to serve as a pedagogical tool
in explaining material for students.

Other researchers have done similar work, an especially
thought-provoking and motivating paper was
\citeauthor{cobb2011teaching}.
The geometric proof of the Gauss-Markov theorem and the introduction of the Herschel-Maxwell approach are inspired by
\citeauthor{cobb2011teaching}'s work.
\citeauthor{jacobson}'s (\citeyear{jacobson}) thesis \citetitle{jacobson}
is a comprehensive overview of a geometric approach in linear models including
statistical foundations.
\citeauthor{gmt_blue}’s (\citeyear{gmt_blue}) text \citetitle{gmt_blue} and
\citetitle{gmt_american_statistician} by
\citeauthor{gmt_american_statistician} (\citeyear{gmt_american_statistician}) were helpful when thinking over
alternative ways of proving this central theorem of econometrics.
The works that deepened our understanding of instrumental variables included
\citeauthor{Butler2016}'s (\citeyear{Butler2016}) \citetitle{Butler2016}
and \href{http://web.hku.hk/~pingyu/6005/6005.htm}{lecture notes} on Econometric theory 1 course by Ping Yu.
The first work that proves the Frisch-Waugh-Lovell theorem using pure geometry
is \citetitle{fwl} by
\citeauthor{fwl} (\citeyear{fwl}).
Finally, there were several works that apply the geometric approach to
hypothesis testing. They are
\citetitle{Langsrud2004}
by \citeauthor{Langsrud2004} (\citeyear{Langsrud2004}),
\citetitle{Siniksaran2005}
by \citeauthor{Siniksaran2005} (\citeyear{Siniksaran2005}),
and one which uses an unusual approach —
\citetitle{friendly2013}
by \citeauthor{friendly2013} (\citeyear{friendly2013}).

The paper has four parts. In the first the fundamentls of the geometry
of random variables are introduced with examples.
In part two, we develop the ideas related to linear regressions
from the ideas which need almost no assumptions to more sophisticated theorems
and concepts.
The third part is devoted to partial correlations. It contains a proof of
a newly-introduced fact about the partial correlation and the correlation
between the residuals in  alinear regression model.
Finally, probability distributions are introduced from a geometric perspective
and hypothesis tests are illustrated.

\end{fullwidth}
