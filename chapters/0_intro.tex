\section{Introduction}

\begin{fullwidth}

Pursuing abstractness and generality, a number of books and articles in
econometrics rely heavily on standard algebraic proofs.
However, most of the theorems prooved in such a way have a strong geometric appeal.
The aim of this paper is to demonstrate how intuitive and elegant the proofs
can be when they are based on well-known geometric facts.
Moreover, we show that the geometric approach can be extended to explain
the concepts in probabily, statistics and machine learning.

Although most of the theorems and ideas are not completely new and
there are even a few works on the geometry in econometrics and statistics,
this paper introduces alternative explanations and more general results in some cases.
Besides, there are algebraic proofs provided parallel to geometric ones
as well as illustrations which are our own work and available on \url{https://github.com/olyagnilova/gauss-markov-pythagoras}. % todo: wiki
They are free to use and have potential to serve as a pedagogical tool
in explaining material for students.

% todo: cite
While preparing the paper we found several works that were useful and relevant.
In particular, Erik D. Jacobson's (2011) thesis \textit{The Geometry of the General Linear Model}
is a comprehensive ovreview of geometric approach in linear models including
statistical foundations.
Leigh J. Halliwell’s (2015) text \textit{The Gauss-Markov Theorem: Beyond the BLUE} and
\textit{An Intuitive Geometric Approach to the Gauss Markov Theorem}
by Leandro da Silva Pereira et al. (2017) were especially helpful when thinking over
the alternative way of proving this central theorem of econometrics.
The works that deepened our understanding of instrumental variables were
Richard J. Butler's (2016) \textit{The Simple Geometry of Correlated Regressors and IV Corrections}
and \href{http://web.hku.hk/~pingyu/6005/6005.htm}{lecture notes} on Econometric theory I course by Ping Yu.
The first work that proves the Frisch-Waugh-Lovell theorem using pure geometry
is \textit{A Geometric Representation of the Frisch-Waugh-Lovell Theorem} by
Walter Sosa Escudero (2001).
Finally, there were several works that apply the geometric approach to
hypothesis testing. They are
\textit{The Geometrical Interpretation of Statistical Tests in Multivariate Linear Regression}
by {\O}yvind Langsrud (2004),
\textit{On the Geometry of F, Wald, LR, and LM Tests in Linear Regression Models}
by Enis Siniksaran (2005),
and the one which uses an outstanding approach —
\textit{Elliptical Insights: Understanding Statistical Methods through Elliptical Geometry}
by Michael Friendly et al. (2013).

The paper consists of five parts. In the first one the fundamentls of the geometry
of random variables are introduced as well as examples.
In the part two, we develop the ideas related to the linear regression
from the ones which need almost no assumptions to the more sophisticated theorems
and concepts.
The next part is devoted to partial correlation. It contains a proof of
a newly-introduced fact about the partial correlation and the correlation
between the residuals in linear regression model.
In part four, the probabily distributions are introduced from the geometric perspective
and the corresponding tests are illustrated.
Finally, there is a part which contains the ideas that can be developed further.

\end{fullwidth}
