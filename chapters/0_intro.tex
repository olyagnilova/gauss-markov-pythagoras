\section{Introduction}

\begin{fullwidth}

Pursuing abstractness and generality, a number of books and articles in
econometrics rely heavily on standard algebraic proofs.
However, most of the theorems proved in such a way have a strong geometric appeal.
The aim of this paper is to demonstrate how intuitive and elegant the proofs
can be when they are based on well-known geometric facts.
Moreover, we show that the geometric approach can be extended to explain
the concepts in probability and statistics.

Although most of the theorems and ideas are not completely new and
there are even a few works on the geometry in econometrics and statistics,
this paper introduces alternative explanations and more general results in some cases.
Besides, there are algebraic proofs provided parallel to geometric ones
as well as illustrations which are our own work. 
The illustrations are published at \url{https://github.com/olyagnilova/gauss-markov-pythagoras} and licensed under the
Creative Commons Attribution 4.0. % todo: wiki
They are free to use and have potential to serve as a pedagogical tool
in explaining material for students.

While preparing the paper we found several works that were useful and relevant.
In particular, \citeauthor{jacobson}'s (\citeyear{jacobson}) thesis \citetitle{jacobson}
is a comprehensive overview of geometric approach in linear models including
statistical foundations.
\citeauthor{gmt_blue}’s (\citeyear{gmt_blue}) text \citetitle{gmt_blue} and
\citetitle{gmt_american_statistician} by
\citeauthor{gmt_american_statistician} (\citeyear{gmt_american_statistician}) were especially helpful when thinking over
the alternative way of proving this central theorem of econometrics.
The works that deepened our understanding of instrumental variables were
\citeauthor{Butler2016}'s (\citeyear{Butler2016}) \citetitle{Butler2016}
and \href{http://web.hku.hk/~pingyu/6005/6005.htm}{lecture notes} on Econometric theory I course by Ping Yu.
The first work that proves the Frisch-Waugh-Lovell theorem using pure geometry
is \citetitle{fwl} by
\citeauthor{fwl} (\citeyear{fwl}).
Finally, there were several works that apply the geometric approach to
hypothesis testing. They are
\citetitle{Langsrud2004}
by \citeauthor{Langsrud2004} (\citeyear{Langsrud2004}),
\citetitle{Siniksaran2005}
by \citeauthor{Siniksaran2005} (\citeyear{Siniksaran2005}),
and the one which uses an outstanding approach —
\citetitle{friendly2013}
by \citeauthor{friendly2013} (\citeyear{friendly2013}).

The paper consists of four parts. In the first one the fundamentls of the geometry
of random variables are introduced as well as examples.
In the part two, we develop the ideas related to the linear regression
from the ones which need almost no assumptions to the more sophisticated theorems
and concepts.
The next part is devoted to partial correlation. It contains a proof of
a newly-introduced fact about the partial correlation and the correlation
between the residuals in linear regression model.
Finally, the probabily distributions are introduced from the geometric perspective
and the corresponding tests are illustrated.

\end{fullwidth}
